\documentclass[]{article}
\usepackage{graphicx}
\usepackage{tikz}
\usepackage{xspace}
\usepackage{enumitem}
\usepackage{mathtools}
\usepackage{amsmath}
\usepackage{float}
%opening
\title{Latency-sensitive Flow management in a Software-Defined Network}

\author{Riccardo Mutschlechner, Stephen Sturdevant, Kausik Subramanian}

\begin{document}

\maketitle

\section{Measurement Model}
In software defined networks, the network is managed by a central controller
and switches can be programmed by  the OpenFlow API by the controller. 
OpenFlow also supports queries to switches, which would retrieve statistics
from the switches. For this project, we can assume a measurement model
where each switch is probed every $t_m$ seconds (the measurement interval)
time. 

Work on this to do presently :
\begin{itemize}
	\item Figure out the different statistics supported by OpenFlow. Primarily, 
	we would need switch queue sizes, and flow statistics on the switch. 
	\item Build a preliminary monitoring system using POX and Mininet. 
\end{itemize}

\section{Estimating Flow Characteristics}
Something similar to CSFQ, ideally we would want to predict the throughput
of a flow with respect to time (denoted by $\Theta$). Look at the next section for decisions based on this. 
This would use the measurement model and $t_m$ and other factors to
estimate the throughput function for the flow. An assumption we can make
in this case is that a flow (identified by a set of headers) will occur 
regularly in the datacenter. This is not unreasonable, consider distributed applications.
For example, short queries can occur between different modules (will possess the
same flow headers), thus estimating a flow can reap benefits in the future as 
we can make better decisions for this particular flow. Another assumption we
can consider is that the throughput function is periodic (?) or based on
some distribution. This needs to be thought about. 

\section{Flow Decisions}
\subsection{Queue buildup model}
Le


\end{document}
