\documentclass[]{article}
\usepackage{graphicx}
\usepackage{tikz}
\usepackage{xspace}
\usepackage{enumitem}
\usepackage{mathtools}
\usepackage{amsmath}
\usepackage{float}
%opening
\title{Latency-sensitive Flow management in a Software-Defined Network}

\author{Riccardo Mutschlechner, Stephen Sturdevant, Kausik Subramanian}

\begin{document}

\maketitle

\section{Measurement Model}
In software defined networks, the network is managed by a central controller
and switches can be programmed by  the OpenFlow API by the controller. 
OpenFlow also supports queries to switches, which would retrieve statistics
from the switches. For this project, we can assume a measurement model
where each switch is probed every $t_m$ seconds (the measurement interval)
time. 

Work on this to do presently :
\begin{itemize}
	\item Figure out the different statistics supported by OpenFlow. Primarily, 
	we would need switch queue sizes, and flow statistics on the switch. 
	\item Build a preliminary monitoring system using POX and Mininet. 
\end{itemize}

\section{Estimating Flow Characteristics}
Something similar to CSFQ, ideally we would want to predict the throughput
of a flow with respect to time (denoted by $\Theta$). Look at the next section for decisions based on this. 
This would use the measurement model and $t_m$ and other factors to
estimate the throughput function for the flow. An assumption we can make
in this case is that a flow (identified by a set of headers) will occur 
regularly in the datacenter. This is not unreasonable, consider distributed applications.
For example, short queries can occur between different modules (will possess the
same flow headers), thus estimating a flow can reap benefits in the future as 
we can make better decisions for this particular flow. Another assumption we
can consider is that the throughput function is periodic (?) or based on
some distribution. This needs to be thought about. 

\section{Flow Decisions}
\subsection{Queue buildup model}
Flows in datacenters require strong latency guarantees, and one of the major
factors causing latency is queue buildup in switches. Thus, it is of paramount
importance to be able to quantify queue buildups in switches to make better scheduling
decisions. By building a queue buildup model, we aim to achieve two things: (1) Provide
latency guarantees (2) Predict congestion. Using the queue buildup model, we can predict 
when latency guarantees or congestion will happen in the future, and accordingly 
decide which flows to move around. 

For a switch $sw_i$ and output port $j$, let us define $F_i^j$ to the be the set of flows 
traversing through the switch to switch $j$ and the link has capacity $C_{ij}$. The Queue
buildup function $\Omega_i^j$ is as follows : 
\[ 
\Omega_i^j(t) = \Omega_i^j(0) + \Sigma_{f \in F_i^j} \int_{0}^{t}\Theta_i^f(t) - C_{ij}t 
\]
The other consideration we need to make is the modifications to $\Theta$ as the
throughput will vary depending on the utilization on each switch the flow traverses
through. The modified throughput function is defined as follows : 
\[
\Theta_j^f(t) = min(C_{ij} \frac{\int_{0}^{\Gamma^f_i}\Theta_i^f(t)dt}
							   {\Sigma_{f \in F_i^j}{\int_{0}^{\Gamma^f_i}\Theta_i^f(t)dt}},
							\Theta_i^f(t))
\]
The modified active time interval of the flow $\Gamma_j^f$ is defined as follows : 
\[
\Gamma_j^f = \frac{\int_{0}^{\Gamma^f_i}\Theta_i^f(t)dt}
							{\Theta_j^f(t)}
\]
The modified inactive interval of the flow $\overline{\Gamma_j^f}$ is as follows : 
\[
\overline{\Gamma_j^f} = \overline{\Gamma_i^f} - (\Gamma_j^f - \Gamma_i^f)
\]

\subsection{Flow Durations}
By building a model of the flow, we can extract two important 
times for improving decision-making in SDNs. We assume flows follow
a ON/OFF model.
\begin{itemize}
	\item $t_{on}$ : The duration of time the flow is active. While this can
	help classify elephant and mice flows; in SDNs there is another use case
	for this. The rules installed by the controllers have a timeout value after which
	the rule is discarded. By estimating accurate time durations, we can set the 
	timeout values so that the rules are removed after the flow becomes inactive,
	thus increasing the efficiency of switch tables.
	\item $t_{off}$ : The duration of time when the flow is OFF. This can be used
	by the controller to pre-emptively install rules. By doing this, the latency-
	sensitive flows do not suffer the additional overhead of the controller
	adding the rules on the switches. 
\end{itemize}


\end{document}
