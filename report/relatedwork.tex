\section{Related Work}
Alizadeh et. al \cite{dctcp} show that long-lived, greedy TCP flows cause the length of the bottleneck queue to grow until packets are dropped, resulting in the familiar sawtooth pattern in buffer-lengths. Queue buildup is also a major factor for increasing latencies observed in datacenters. To minimize queue buildup, they develop a modified TCP, DCTCP which uses the queue statistics at switches for congestion control, nearly achieving zero queue buildup. 

Flow scheduling in datacenters has been an active area of research in recent times. Hedera \cite{hedera} performs scheduling of flows by estimating TCP bandwidth requirements and dynamic flow scheduling using different placement algorithms (most importantly simulated annealing which is a probabilistic search technique for paths) to achieve full bisection bandwidth requirements. On the other hand, we focus on the problem of dynamically scheduling flows in terms of latency-sensitivity (mice flows, unlike elephant flows, are highly sensitive to latency), which also requires monitoring queue sizes at switches in real-time and a dynamic path placement with the objective of decreasing latency. Hedera does not focus on the aspect of latency in its flow-scheduling algorithm. This also reduces the timescales of scheduling decisions, and requires the need to devise real-time flow-placement algorithms which need not be optimal. 

Fastpass \cite{fastpass} presents a datacenter network architecture where the time of packet transmissions and the path traversed by packets are determined by a centralized arbiter. This approach uses a timeslot allocation algorithm to determine when the endpoint’s packets should be sent and a path assignment algorithm to determine paths to eliminate congestion at switches. The centralized arbiter ensures zero queue buildup in the switches by globally scheduling each packet, and thus flows do not suffer queueing delays. However, there is a throughput penalty for achieving very low queueing delays, as some links would have to be idle. 
 
 Crovella et. al \cite{CFH99} examined a commonly used service, Apache; specifically, its routing policies. The paper points out that Apache does not prioritize shorter requests versus longer requests. The authors then examine the potential benefits of different queueing algorithms that prefer shortest-connection first, whose principles can be used in the context of flow-scheduling. The paper mentions that web server workloads are primarily tail-heavy, which may work to our benefit in the sense that elephant flows will not pay much of a penalty relative to the gains that mice flows will see. 