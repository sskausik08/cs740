\section{Motivation}
To frame our problem more specifically, we define two notions of flows: mice flows, and elephant flows. Mice flows are low-throughput, latency-sensitive flows which are generally used for distributed applications and are tightly coupled with application performance. These types of flows have very low tolerance to loss and latency. Secondly, we have elephant flows, which are high-throughput, latency-tolerant flows. These flows are generally used for data replication and MapReduce style jobs. 

When elephant and mice flows traverse the same queue, the major reason for increased latency is queue buildups. This is very problematic for the latency-intolerant mice flows. There are two primary cases shown in Figure \ref{fig:cases} in which queue buildup severely increases the latency of mice flows. In the first case, the mice flows experience increased latency as they are intermixed with the elephant flows in queue, getting an unfair share of transmission time as well as high latency and jitter. In the second case, the mice flows are being enqueued at the tail of a queue which is already completely backed up by existing elephant flows. This is also very problematic, as the mice flow packets will either be very highly delayed or dropped altogether. Thus, there is a need to classify flows and monitor switch queues to detect queue buildups in real-time. When we detect queue buildups, we need to dynamically schedule flows on different routes based on their classification to ensure low latencies. 

\begin{figure}[H]
\centering
\begin{subfigure}{.55\textwidth}
  \centering
  \includegraphics[width=.8\linewidth]{case1}
  \caption{Case 1}
  \label{fig:case1}
\end{subfigure}%
\\
\begin{subfigure}{.55\textwidth}
  \centering
  \includegraphics[width=.8\linewidth]{case2}
  \caption{Case 2 }
  \label{fig:case2}
\end{subfigure}
\caption{Cases of Queue Buildup}
\label{fig:cases}
\end{figure}

The prominence of software-defined networks has increased in recent times, and offers a centralized control of the network. The SDN controller has a global view of the network (such as topology and routing information) and can collect any relevant statistics (such as link utilization and traffic matrix). Using the resources available in a typical SDN setup, we believe there exist the means to schedule flows in real-time to avoid the problem of mice and elephant flow conflicts. Thus, we ask the following question: \emph{How can we perform dynamic flow scheduling to minimize latency of the mice flows in a software-defined datacenter? }  
