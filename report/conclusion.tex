\section{Conclusion and Future Work}
In this project, we present a predictive queue management system for software-defined
networks, and demonstrate the best-case performance improvement of the approach.  

One of the concerns with this approach is its scalability. Since the measurement and scheduling
is independent from each other, distributed controllers can be used for measuring slices 
of the network, and can be aggregated to a centralized server for making the scheduling decisions.
Similarily, the new rules to be installed can be distributed across the controllers for fast installation.
However, the scheduler can be a bottleneck as number of flows increase, and the scheduling epoch
has to be decided in conjuction of the scale of the network. Also, another direction of research is 
to evaluate the optimal scheduling (instead of the greedy approach proposed in this paper). 

By building predictive models of flows, there are interesting applications to this approach. This can
be used for predicting good rule timeout values for switches, and perform efficient rule management
in SDNs. Also, the predictive models can be use to proactively install rules in the network, thus reducing
the latency suffered by flows due to forwarding rules installation. 

Finally, this project helped us understand and build models characterizing different phenomena
in the networks, and the complexity of dependent phenomena. The experience of building a complete
system from scratch was fulfilling, and made us realize the challenges involved in them. Finally, software-
defined networks are still in a nascent phase, and our failed experiments with Mininet and non-availability of 
good simulation frameworks like \emph{ns-2} emphasise the requirement of good prototyping software
for SDN applications to run simulations and evaluate different parameters. 